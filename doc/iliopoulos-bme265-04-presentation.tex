\documentclass[12pt,x11names]{beamer}
\usetheme{Copenhagen}


%%%%%%%%%%%%%%%%%%%%%%%%%%%%%%%%%%%%%%%%%%%%%%%%%%
%%% PREAMBLE


% -------------------- packages -------------------- %

% page formatting
% \usepackage[margin=1in]{geometry}
\usepackage{setspace}

% math
\usepackage{amsmath}
\usepackage{amsfonts}
\usepackage{amsthm}
\usepackage{mathrsfs}
\usepackage{array}
\usepackage{cancel}
% \usepackage{breqn}

% floats
\usepackage[section]{placeins}
% \usepackage{float}
% \usepackage{listings}
% \usepackage[section]{algorithm}
% \usepackage{algpseudocode}
\usepackage[font={small,color=darkgray},labelfont=bf,hypcap=true]{caption}
\usepackage[font={footnotesize,color=darkgray},labelfont=bf,hypcap=true]{subcaption}
% \usepackage{sidecap}

% graphics
\usepackage{graphicx}
% \usepackage[pdftex,hyperref,x11names]{xcolor}
% \usepackage{tikz}
% \usepackage{pgfplots}

% tables
\usepackage{multirow}
\usepackage{booktabs}

% lists
\usepackage{enumerate}
\usepackage{paralist}

% header and footer
\usepackage{fancyhdr}

% footnotes
\usepackage[bottom]{footmisc}

% ordinal numbers
\usepackage[super]{nth}

% fonts
\usepackage[resetfonts]{cmap}

% references
\usepackage[numbers,square]{natbib}
% \usepackage[pdftex]{hyperref}
% \usepackage{cleveref}

% -------------------------------------------------- %


% -------------------- customizations -------------------- %

% define coloured clickable links
\hypersetup{%
pdffitwindow=false,%
colorlinks,%
pdfauthor={Alexandros-Stavros Iliopoulos},%
pdftitle={Image registration in the Radon domain},%
pagebackref=false,%
citecolor=PaleGreen4,%
filecolor=DarkOrchid4,%
linkcolor=OrangeRed4,%
urlcolor=RoyalBlue4%
}

% define the includegraphics search path
\graphicspath{{figs/}{figs/pdf/}{../code/}}

% easy command for boldface math symbols
\newcommand{\mbs}[1]{\boldsymbol{#1}}

% define a dark gray color to be used in captions
\definecolor{darkgray}{rgb}{0.25,0.25,0.25}

% define bibliography style
\bibliographystyle{plain}

% -------------------------------------------------------- %


% -------------------- title -------------------- %

\title{\textbf{Image registration in the Radon domain}}
\author[A. S. Iliopoulos]{Alexandros-Stavros Iliopoulos\\
  \href{mailto:ailiop@cs.duke.edu}{\texttt{ailiop@cs.duke.edu}}}
\institute[Duke University]{Duke University}
\date{\today}

% ----------------------------------------------- %



%%%%%%%%%%%%%%%%%%%%%%%%%%%%%%%%%%%%%%%%%%%%%%%%%%
%%% DOCUMENT


\begin{document}

\def\newblock{\hskip .11em plus .33em minus .07em}


\begin{frame}
\titlepage
\end{frame}


\begin{frame}
  \frametitle{Foreword}
  
  The project is a new-born.

\end{frame}


\begin{frame}
  \frametitle{What flavor of registration?}

  \begin{itemize}

  \item The proposed methodology aims to register frames whose relative
    alignment can be described through \emph{affine} geometric
    transformations:
    \begin{itemize}
    \item Reflection
    \item Scaling
    \item Translation
    \item Rotation
    \end{itemize}

  \item Motion is not required to be of sub-pixel magnitude.

  \item Computations occur in the discrete Radon domain.
    \begin{itemize}
    \item The model is based on~\cite{hjouj_identification_2008}.
    \end{itemize}

  \end{itemize}
\end{frame}


\begin{frame}
  \frametitle{Advantages}
  
  \begin{itemize}

  \item Can be robust to noise.

  \item E.g.\ let $g = f + \eta$, where $\eta$ is zero-mean, white Gaussian
    noise:
    \begin{align*}
      E\{ \mathfrak{R}g(p,\phi) \}
      &= E\left\{ \int_L{g(x,y) dl} \right\} \\
      &= E\left\{ \int_L{[f(x,y) + \eta(x,y)] dl} \right\} \\
      &= E\left\{ \int_L{f(x,y) dl} \right\}
      + \cancelto{0}{E\left\{ \int_L{\eta(x,y) dl} \right\}} \\
      &= E\{ \mathfrak{R}f(p,\phi) \}
    \end{align*}
    \begin{itemize}
    \item Also true for the discrete case.
    \end{itemize}
  \end{itemize}

\end{frame}


\begin{frame}
  \frametitle{Advantages (Cont'd)}
  
  \begin{itemize}

  \item Not all information in the Radon domain are ``required.'' (E.g.\
    without rotation, 1 column of the Radon transforms of the frames
    suffices.)
    \begin{itemize}
    \item Further opportunities for noise tolerance.
    \item Compression.
    \end{itemize}

  \item Computations are done via (relatively) simple vector operations.
    \begin{itemize}
    \item Efficiency?
    \end{itemize}

  \end{itemize}
\end{frame}


\begin{frame}
  \frametitle{Goals}
  
  \begin{itemize}
  \item Measure and compare registration in the Radon domain with other
    registration approaches (e.g.\ Fourier-based methods).
    \begin{itemize}
    \item Performance.
    \item Robustness to noise.
    \item Efficiency/complexity.
    \end{itemize}
  \end{itemize}
\end{frame}

\nocite{*}


%%%%%%%%%%%%%%%%%%%%%%%%%%%%%%%%%%%%%%%%%%%%%%%%%%%%
%%% REFERENCES

\bibliography{refs-presentation.bib}


\end{document}
